\begin{document}
%=================================================================
%                           Start Document
%=================================================================
\section{Introduction }
\lhead{Introduction} % section header

\setstretch{1.6}

\subsection{Motivation}

Several conditions may cause lower limb motor disorders including stroke, cerebral palsy and muscular dystrophy \cite{hayami_development_2022}. There are around 101 million people living with the effects of stroke \cite{noauthor_key_nodate}, nine of ten of which experience gait disorders \cite{schaechter_motor_2004}. Hemiparesis, which is a condition characterized by weakness or partial paralysis on one side of the body effect 65\% of stroke victims \cite{wist_muscle_2016} and is a primary contributor to gait disorders. Losing the ability to move voluntarily can have decastating consequences for both the independence and quality of life of the person affected \cite{marquez-chin_functional_2020}. There is therefore a large need for effective rehabilitation techniques for these patients. This project implements a system aimed at improving rehabilitation specifically for stroke patients struggling with hemiparetic gait disorders, however the control systems are applicable to the treatment of neurological motor disorders in general.

Today there are several options for gait rehabilitation, the conventional treatment is physiotherapy, however Functional Electrial Stimulation (FES) has become more popular with time \cite{muller_adaptive_2020}. When electrical stimulation is used along with functional training to facilitate movement performance (such as relearning to walk), it is referred to as FES training \cite{hayami_development_2022}. It has been demonstrated that FES can improve blood circulation, range of motion, muscle strength and muscle spasticity \cite{luo_review_2020}. FES stimuation of muscles and nerves can also help restore communication pathways between brain and muscle, improving neuroplasticity essential for movement recovery \cite{marquez-chin_functional_2020}. In order to improve gait rehabilitation a push should therefore be made to improve the current FES stimulation strategies in order to make them more effective and adaptable.

Most existing FES systems on the market still rely on open-loop systems, mostly due to their simplicity\cite{braz_functional_2009}. However this typically requires continuous or repeated user input (ususally via a device button) for muscle activation, and consequently requires a lot of attention \cite{hayami_development_2022}. This method is therefore not only burdensome and prone to human error but also may be dangerous due to the risk of falls. Another major limitation is that it lacks adaptability in stimulation intensity, potentially leading to inefficient stimulation patterns, overstimulation, increased fatigue and discomfort. The human muscular system is inherently non-linear and time-dependent with factors such as fatigue, voluntary interaction, spasticity and muscle habituation causing constantly changing muscular responses \todo{source}. As a result open-loop stimulation is highly suboptimal, and closed-loop systems capable of real-time adaptation should be considered in order to improve safety, efficiency and user comfort.

Two types of closed-loop control systems are proposed here. The first system relies on joint angle feedback and a preset stimulation sequence. It dynamically adjusts stimulation intensity based on the deviation from well-known reference joint angles. The preset stimulation sequence, although inefficient for a chronic device could work well for the treadmill-based gait training this project aims to improve. This relatively simple form of closed loop would already be large improvement reducing the high attention burden on users, minimizing overstimulation and lowering the risk of falls.  

The second closed-loop system builds upon the first by incorporating an online gait phase detection algorithm, which would allow for more precise timing of gait phase transitions, more-so mimicing the nervous system's natural control of gait. This approach is an improvement upon the first system, as well as traditional open-loop implementation by allowing the stimulation sequence to adapt dynamically rather than being preset. This would further enhance safety and be more efficient, particularly in patients with some voluntary muscular control.

\subsection{Background}

\subsubsection{Hemiparetic Stroke and the Role of FES}

Hemiparetic stroke, a common result of cerebrovascular accidents, often causes motor impairments on one side of the body, disrupting the coordination needed for normal gait. These impairments manifest as asymmetry, reduced speed, and inefficiency, severely limiting mobility and quality of life. Contributing factors include muscle weakness, spasticity, and loss of motor control.

Functional Electrical Stimulation (FES) uses electrical currents to activate motor neurons and stimulate muscle contractions in individuals with impaired voluntary control. In hemiparetic stroke, FES is often used to address foot drop by stimulating the tibialis anterior during the swing phase, improving dorsiflexion and reducing compensatory movements. This leads to more efficient and symmetrical gait patterns.

Beyond immediate gait correction, repeated FES during training can promote neuroplasticity, enhancing voluntary motor control by reinforcing motor learning and cortical reorganization. FES also strengthens muscles, reduces spasticity, and improves cardiovascular fitness, contributing to overall recovery.

While FES offers significant benefits, challenges include muscle fatigue, discomfort, and the need for individualized parameter calibration based on stroke severity and anatomical differences. Despite these limitations, FES is a promising tool for restoring mobility, independence, and quality of life in individuals with hemiparetic stroke.

\subsubsection{FES Control Strategies}
Functional Electrical Stimulation has been researched over several decades as a means to restore motor function in individuals with neurological impairments. Control strategies for FES can be broadly categorized into open-loop, linear and nonlinear approaches. However, closed-loop control has shown to be clearly the most promising method for addressing the complexities of muscular biomechanics \cite{chaikho_transcutaneous_2022}, particularly for gait.

In closed-loop FES stystems, the manipulated variables are the stimulation wave parameters for intensity, current amplitude and pulse width. These parameters are adjusted to control either neuromuscular variable (e.g. muscle force), kineamtic variables (e.g. joint angles and angular velocities) or kinetic variables (e.g. joint torques) \cite{chaikho_transcutaneous_2022}.

Linear control strategies, such as Proportional-Integral-Derivative (PID) controllers, are frequently implemented in FES systems due to their simplicity and effectivenesss in controlled environments (\cite{chaikho_transcutaneous_2022}, \cite{bouri_closed-loop_2018}, \cite{dodson_experimental_2017}). These controllers regulate the stimulation parameters to achieve predefined kineamtic or kinetic targets. These controllers may be combined with gait phase detection, allowing phase-specific modulation of stimulation. Despite the applicability of linear control, it has several limitation when applied to FES-induces movements. Their primary drawback is that they are theoretically not optimmal for addressing the nonlinear and time-varying dynamics of the musculoskeletal system. Consequently several groups have attempted to combine FES with nonlinear control.

The nonlinear approaches applied to FES include fuzzy logic control (FSC), sliding mode control (SMC), predicitive control (PC) and adaptive control \cite{chaikho_transcutaneous_2022} \todo{source for each}. However none of these, as per the knowledge of the writer have been applied to FES for gait. However these control strategies, particularly adaptive control, which continuously updates parameters based on real-time feedbaback is promising. This would allow the system to compensate for both the time-varying dynamics and the individual variations in muscular responses. However the complexity and possible computational demands remain a challenge. 

In several sudies \todo{examples} dynamic models are used to establish a relationship between the FES inputs and the outputs. This is done because such models allow for a predictive understanding of the impact of the stimulation, leading to more robust and accurate control strategies. However these models are complex, and have never (as per the knowledge of the writer) been used for a full gait implementation. The current models \todo{examples} are mostly suited to modelling a single muscle in a controlled environment. In the case of gait the models would need to require consideration of the dunamics of muscle activation, the dynamics of contraction and the biomechanical model of gait \cite{chaikho_transcutaneous_2022}. It would also have to mathematically model a system with several unique parameters per individual. This leads to no single model being applicable to gait FES, however one could expect that future work might focus on creating more black box models using neural networks to take on this task.

Despite the advancement in FES control strategies, the application of nonlinear control and dynamic models to gait rehabilitation reamains very limited. Most studies employing more complex control strategies focus on single-muscle stimulation, and very few of them utilize IMUs for feedback. Furthermore, models capable of capturing the dynamics of full gait do not yet exist. Resulting in the conclusion that for gait rehabilitation the most applicable solution remains to be linear control with possible enhancements by incorporating model-free adaptive control.

\subsection{Project Interpretation}
This project aims to develop a closed-loop functional electrical stimulation control system for gait rehabilitation following hemiparetic stroke. To that end several time consuming steps needed to be completed before moving on to direct closed loop design. Despite the broad scope, the primary goal was to produce reliable, well-documented work grounded in established theory and literature. This approach ensures that the project can serve as a solid foundation for further development and eventual clinical application.

Given the time constraints and the scope of tasks, a simple linear control system was designed and implemented. The goal was to establish a simple first version of the closed loop system that could simply be expanded by for example adding adaptive elements or new algorithms for stimulation transition later on. The ultimate aim being to lay a strong groundwork for future work in the lab, rather than rushing to implement a more complex system that might later require significant revisions, as seen in a previous project that this project was intended to build upon.

This project also implements a control system for stimulation of one leg only at a time. This is due to the goal of improving the rehabilitation following hemiparetic stroke, meaning that only half of the body is partly paralized. This can however later be adjusted to work for both legs for applications such as spinal cord injury.



\subsection{Project Context}

This project builds upon an ongoing effort at the REHassist lab at EPFL for developing hybrid strategies using robotics and Functional Electrical Stimulation (FES) for treating neurological motor disorders such as stroke, cerebral palsy and spinal cord injury. 

Importantly this project uses hardware previously developed at the REHassist lab. Firstly there is the StimWave3 which is a custom, 20 channel, electrostimulator capable of generating continuous or pulsed currents with adjustable frequency, pulse width, and amplitude. Secondly there are the wireless IMUs that output angular velocity and acceleration along three axis. 

Software wise, there is a codebase that has been developed for hybrid FES robotics strategies using the StimWave hardware. Specifically there is a codebase that was used for FES with a Leg-Press robotic trainer called LegoPress \cite{olivier_legopress_2014}. That code was adapted last semester by another student for the closed loop gait application. A goal of this project is to further develop the codebase. However, a part of that must include refactoring of this code. This is because the closed and open loop control logic was implemented on top of the legopress logic and everything was effectively implemented in one file, leading to poor code quality. This is further expanded upon in the implementation section on refactoring.

The project on closed loop gait application mentioned above was also intended to lay some of the groundwork for this project. However upon starting it became clear that the code, open loop design, closed loop design and implementation would need to be rethought and implemented in a new manner.



\subsection{Contributions}

This project resulted in several contributions to the furthering of closed-loop FES for gait rehabilitation, particularly within the REHassist lab. A modualr and maintianable code base was established through comprehensice refactoring of legacy software, ensuring sclability and reausability. The project introduced a robust methodology for determining stimulation parameters, aligning them with clinical standards.

A key contribution is the design and validation of the open-loop stimulation sequence capable of replicating a step for a range of users. This sequence was tested on multiple subjects, demosntrating generalizability and creating the basis for further closed-loop implementation. Wireless inertial measurement units (IMUs) were also successpully integrated. Several orientation estimation methods were evaluated and an adapted Madgwick filter was implemented in order to estimate knee angles accurately, enabling real-time feedback control.

The closed-loop control system, was designed to dynamically adjust stimulation based on knee angle feedback. A time-dependent reference curve, derived from Fourier series fitting to knee angle data during healthy gait was implemented to guide the controller. This system was tested at different walking speeds, demonstrating the potential to improve stimulation efficiency and reduce fatigue and discomfort.

Beyond the technical contributions, this report provides a transparent analysis of all implemented aspects, highlighting their strengths while identifying areas for improvement and offering suggestions for future work. This approach establishes a solid foundation for further development and progress towards a reliable framework for eventual clinical application.


\subsection{Outline}
This work is structured into first this introduction which highlights the motivation behind the research and the project scope, objectives, and contributions are outlined. Thereafter there is a background section that explores the principles of FES and gait mechanisms.

The main body of the thesis is divided into six core topics, each addressing a specific task. These sections generally include methods, implementation, results, and discussion, with some also featuring design and further work subsections.

The first topic is software refactoring, detailing the restructuring and development of the software framework. The second is the functional electrical stimulation setup, which describes the design and implementation of stimulation parameters, including waveform, frequency, and intensity. The third section focuses on the open-loop stimulation sequence, examining the limitations of the initial sequence, the design of a new sequence, and its validation. The fourth topic, knee angle estimation, discusses the integration of wireless IMUs and methods for estimating joint angles, alongside the results of this implementation. This is followed by a section on gait phase detection, which covers the attempted implementation of an online detection algorithm. The final section addresses closed-loop FES control, describing the design, implementation, and evaluation of the closed-loop system, along with an interpretation of the results.










%==============================================================
%                           End Document
%==============================================================
\end{document}