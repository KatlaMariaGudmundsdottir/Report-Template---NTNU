\begin{document}
%=================================================================
%                           Start Document
%=================================================================
\section*{Abstract}
Motor impairments caused by hemiparesis, a common consequence of stroke, affect millions of people worldwide  \cite{noauthor_key_nodate}. The resulting inability to walk or move independently places a significant physical, emotional, and social burden on patients, driving a strong demand for effective rehabilitation solutions \cite{noauthor_physical_nodate}. Functional Electrical Stimulation (FES), a technique that applies electrical impulses to activate muscles, has shown promise in enhancing motor function, promoting neuroplasticity, and supporting recovery \cite{luo_review_2020}. Despite its potential, most existing FES systems rely on open-loop control \cite{braz_functional_2009}, which lacks the adaptability, efficiency, and safety required for optimal rehabilitation outcomes.

This project addresses these limitations by developing a closed-loop FES control system for gait rehabilitation. Leveraging real-time feedback from inertial measurement units combined with linear stimulation control combined and a feedforward element. This design ensures safer, and more efficient gait training by dynamically adjusting stimulation intensity based on deviations from reference joint angles, thereby reducing user burden and minimizing overstimulation.

Key contributions beyond the implementation of the closed-loop control include the design of an open-loop stimulation sequence, integration of wireless IMUs and the development of a sensor fusion algorithm for real-time joint angle estimation. All implemented within a modular and scalable software framework. The system and its components were systematically validated through testing, demonstrating its potential to improve stimulation efficiency and thereby rehabilitation outcomes. This work lays a foundation for advancing closed-loop FES systems, with the goal of refining rehabilitation techniques to create meaningful improvements in patients’ lives.

\setstretch{1.6}
%================================================================
%                           End Document
%================================================================
\end{document}