\begin{document}
%=================================================================
%                           Start Document
%=================================================================
\section{Further work}
\lhead{Further Work} % section header

\setstretch{1.6}
\subsection{Open Loop Stimulation Sequence}

In future studies, incorporating a weight support system could help isolate the effects of FES on gait mechanisms and create a more controlled environment, providing a clearer assessment of its efficacy.

The findings, especially the observation of the gait falling out of sync with stimulation at times, underscores the limitations of open-loop systems and the potential advantages of closed-loop systems with real-time phase detection. By dynamically adjusting stimulation based on the current phase the consistency and functionality of the FES bwould be improved, resulting in a more natural and effective gait restoration

\subsection{Knee Angle Estimation}
One of the challenges encountered when implementing a knee angle estimation algorithm is the need for alignment of the sensor frames to the relevant body segments. Although this is in effect implemented in the madgwick filter by .... 


Future work could explore the integration fo real-time alignment correction mechanisms, potentially leveraging additional data sources, such as magnetometers to monitor and adjust sensor alignment continuously.

Further work could involve implementing motion segmentation algorithms to differentate between periods of high and low dynamic activity allowing the algorithm to weigh accelerometer corrections more effectively.

not implemented here a possible further development would be to implement a sensor-to-segment alignment. Such algorithm can generally be categorized into manual alignment, static pose estimation, functional calibration, deep learning, or some combination of these approaches \cite{rhudy_knee_2024}. For further developing this project a simple static pose such as holding still for 5 to 10s in a specified standing pose could be utilized. The accelerometers would be utilized to detect the gravity vector and thus 


\subsection{Closed-Loop}
Future studies would benefit from adding a weight support system to minimize voluntary reqruitment when testing in healthy individuals. However testing in a in indiciduals with neurological impairments would clearly be the best case scenario. 

The limited subject pool in this project is another issue that should be addressed in future work.

discuss that in further iterations it would be interesting to use a reference curve that was averaged acroos multiple subjects in order to make it more generalizable, one could even make specific demographic groups and speed groups such as female young adult or male middle aged, since there is some variation in gait dynamics between demographics, especially related to age and sex and weight

%=================================================================
%                           End Document
%=================================================================
\end{document}