\begin{document}
%=================================================================
%                           Start Document
%=================================================================
\section{Implementation}
\lhead{Implementation} % section header
\setstretch{1.6}

\subsection{Software Refactoring}
As mentioned introduction wise this project builds upon the work and code base originally developed for FES with the LegoPress \cite{olivier_legopress_2014} and further developed by a previous student. However the code quality was poor with thousands of lines of code and multiple functionalities all implemented in one file and one class. Therefore before continuing on with the project a proper refactoring of the code was necessary.

In order to understand the code it was decided that the first step would be to create documentation, specifically graphical representations of the interactions and hierarchy. To accomplish the code was documented and edited so that it would be compatible with doxygen. Doxygen is a documentation generation tool that automatically creates software documentation from annotated source code in HTML. 



As mentioned introduction wise this project builds on a code base developed at the REHassist lab. 

\subsection{Functional Electrical Stimulation Sequence}

\subsection{IMU Integration}

\subsection{Knee angle Extraction}

\subsection{Phase Detection}

\subsection{Closed loop}

\subsection{Functional Electrical Stimulation}

\subsubsection{Stimulation device: StimWave3}

\subsection{Closed Loop}

\subsubsection{Feedback: Inertial Measurement Unit}



gg
%=================================================================
%                           End Document
%=================================================================
\end{document}